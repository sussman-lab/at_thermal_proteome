Proteins are fundamental macromolecules involved in all aspects of life, from
catalyzing metabolic reactions to providing a scaffold for cellular
organization to transmitting external environmental changes into the nuclear
transcriptional machinery. Until recently, nearly all large-scale proteomic
studies have focused on quantifying changes in protein abundance or degree of
post-translational modification to amino acid sidechains. In reality, however,
changes in protein function result from changes in conformation at the
secondary, tertiary, and higher-level structures. Indeed, current evidence
suggests that three-dimensional structure is more highly conserved between
evolutionarily related proteins than is their primary amino acid sequence
\cite{cramer_structural_2001,shih_bacterial_2006,ingles-prieto_conservation_2013}.
Unfortunately, the technology to examine three-dimensional structure at the
proteome scale has historically been lacking. Recently, several technologies
have emerged that attempt to address this deficiency and have been applied to
animal studies, but to our knowledge no such studies have yet been published
in the domain of plant research.

It is widely accepted that most cellular interactions involving proteins
depend upon and/or induce changes in a protein's three-dimensional
conformation. The kinetics and energetics of these changes are closely related
to the protein's thermal stability. It is thus reasonable to expect that an
organism's proteome will have evolved to minimize the energy needed to
maintain any given protein's function at physiological temperature while
taking into account the requirements of any modes of regulation it may
undergo. An understanding of global differences in relative thermal stability
of proteins may thus provide insight into how different proteins have evolved
to function in an energetically efficient way. In addition, given that many
plants are exposed to the elements in all seasons and experience large
fluctuations in temperature, it is logical to ask whether their proteomes have
developed unique characteristics of thermal stability and conformation to
preserve their functions under changing environmental conditions.

There are numerous quantifiable attributes of a protein potentially related to
its conformation and thermal behavior. One such characteristic is the melting
temperature (\Tm{}), typically defined as the temperature at which half of a
protein population is unfolded. Until very recently, available techniques for
estimation of protein \Tm{} have relied upon measurement of various properties
of a purified protein solution \textit{in vitro}. This generally involves
isolating a purified protein of interest and observing changes in physical or
chemical properties of the solution across a temperature gradient.
High-throughput screens are also possible --- for instance, when the
measurement technique can be carried out in 96- or 382-well plates --- but
this still requires purification of individual proteins, a laborious and
time-consuming step.

In 2014, an untargeted method called thermal proteome profiling (TPP) was
introduced which used isotope-encoded multiplexed mass-spectrometry (MS)-based
quantification to profile thousands of proteins simultaneously
\cite{savitski_tracking_2014}. The basis of the technique is similar to those
mentioned above, in that a measure of protein conformation is tracked across a
temperature gradient (Fig.\ \ref{fig:intro}). In this case, protein solubility
is used as a proxy for folding state given that unfolded, denatured proteins
precipitate out of solution. After centrifugation to remove proteins
precipitated across a temperature gradient, MS/MS of isobarically labeled
tryptic peptides derived from the remaining non-denatured proteins is used to
measure relative abundance of individual proteins. The resulting
temperature--abundance profile for each protein is fit to a standard two-state
protein melting model and used to calculate a \Tm{} (Fig.\ \ref{fig:model}).
Since the original TPP protocol was published, additional methodologies have
been introduced which utilize different readouts for protein stability and/or
different MS-MS based quantification strategies. In the work described herein,
we have chosen to utilize multiplexed isobaric labeling as in the original TPP
paper because of the ease of direct relative quantification at all temperature
points for every peptide identified.

There exists considerable untapped potential for TPP in the plant research
community, in which many receptor-ligand pairs and protein-protein
interactions remain poorly understood. To that end, we have undertaken a
characterization of the thermal proteome of the model plant
\textit{Arabidopsis thaliana}, of which much is already known about the
proteome and its modifications. By using a relatively large number of
biological replicates and applying extensive offline fractionation, our aim
was to produce a large and robust database of untreated \Tm{} measurements
that could serve as the groundwork for future targeted work in the species. In
particular, it is critical to understand the limitations of a technology in
the domain of interest, and the database of melting profiles developed in this
work provides a valuable resource of information on which proteins ``behave
well'' in the assay and with what experiment-to-experiment variability,
providing researchers with information on their proteins of interest prior to
embarking on a TPP experiment.

In the course of this work, the question also arose whether the database of
\Tm{} measurements could be used to add to the understanding of more general
questions regarding protein structure and thermostability. Such questions are
of widespread interest both to basic researchers and to those interested in
applying such knowledge to the engineering of novel proteins. To this end, we
undertook an analysis of the correlation between the empirical \Tm{}s and
various possible physicochemical determinants of protein thermostability. In
particular, we were interested in how such determinants might be preserved or
differ between a ``poikilothermic'' plant proteome and existing data from
other kingdoms.

Lastly, we demonstrated the ability, in a complex extract from plant tissue,
to detect \textit{in vitro} conformational changes at the proteome-wide scale
caused by a common co-factor, adenosine triphosphate complexed with Mg2+, and
correlate these changes with existing knowledge of protein binding sites.
Taken together, this initial work establishes a baseline for future studies on
wild-type and mutant Arabidopsis and other plant species grown under a variety
of environmental, chemical and genetic perturbations.
