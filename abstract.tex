%max 300 words
Modern tandem MS-based sequencing technologies allow for the parallel
measurement of concentration and covalent modifications for proteins within a
complex sample. Recently, this capability has been extended to probe a
proteome's three-dimensional structure and conformational state by determining
the thermal denaturation profile of thousands of proteins simultaneously.
While many animals and their resident microbes exist under a relatively
narrow, regulated physiological temperature range, plants take on the often
widely ranging temperature of their surroundings, possibly influencing the
evolution of protein thermal stability. In this report we present the first
in-depth look at the thermal proteome of a plant species, the model organism
\textit{Arabidopsis thaliana}. By profiling the melting curves of over 1700
Arabidopsis proteins using six biological replicates, we have observed
significant correlation between protein thermostability and a number of known
protein characteristics, including molecular weight and the composition ratio
of charged to polar amino acids. We also report on a divergence of the
thermostability of the core and regulatory domains of the plant 26S proteasome
that may reflect a unique property of the manner in which protein turnover is
regulated during temperature stress. Lastly, the highly replicated database of
Arabidopsis melting temperatures reported herein provides baseline data on the
variability of protein behavior in the assay. Unfolding behavior and
experiment-to-experiment variability were observed to be protein-specific
traits, and thus this data can serve to inform the design and interpretation
of future targeted assays to probe the conformational status of proteins from
plants exposed to different chemical, environmental and genetic challenges.
